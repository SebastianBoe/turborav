\documentclass[12pt]{article}

\usepackage{listings}
\usepackage{todonotes}

\begin{document}
\section{Guidelines}
\subsection{Version control}
\begin{enumerate}
\item Merge commits should be avoided.\footnote
  {
    They add no value, can be avoided through proper use of rebase, and
    make looking at the linear history of commits to Master confusing.
  }
  \item When committing something non-trivial to code with a clear
    original author it is considered polite to do a pull-request.\footnote
    {
      On the pro-side this introduces some code review and allows the
      original author to stay up-to-date with the state of code he cares
      about. On the con-side this increases the ``terskel'' for improving on
      code and can cause improvements to be dropped.
    }
\end{enumerate}
\subsection{Source code}
Unless stated otherwise one must follow the style guide of the
language.
\begin{enumerate}
  \item No trailing-whitespace.
  \item No tabs.
  \item Max line length is 80 chars.
\end{enumerate}
\subsubsection{Chisel}
\todo[inline]{It might increase readability if we had the policy of using
  functions for combinatorial HW, and Modules for stateful HW.}

A chisel source file should follow the structure of the below code.\footnote
{
  Being consistent in the structure of each source file increases readability.
}
\begin{figure}
\begin{lstlisting}[language=Java, frame=single]
package Turborav

import Chisel._

// Some high-level description of what the module does.

class MyModule(implicit conf: TurboravConfig)
extends Module {
  // start with io.
 val io = new Bundle {
   val in  = Bool(INPUT)
   val out = Bool(OUTPUT)
 }
  // Instantiation assertions
  require(Array(32, 64) contains conf.xlen)

//Chisel code.

 io.out := false
}
\end{lstlisting}
\end{figure}
\end{document}
